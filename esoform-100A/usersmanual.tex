%%%%%%%%%%%%%%%%%%%%%%%%%%%%%%%%%%%%%%%%%%%%%%%%%%%%%%%%%%%%%%%%%%%%%%%%
%
%.IDENTIFICATION $Id: usersmanual.tex.src,v 1.25 2007/09/04 14:53:32 fsogni Exp $
%.LANGUAGE       TeX, LaTeX
%.ENVIRONMENT    ESOFORM
%.PURPOSE        ESOFORM User manual
%.AUTHOR         This file has been updated by the Observing Programmes Office (OPO)
%
%%%%%%%%%%%%%%%%%%%%%%%%%%%%%%%%%%%%%%%%%%%%%%%%%%%%%%%%%%%%%%%%%%%%%%

\documentclass{article}
\topmargin=-1cm
\textheight=23.7cm
\textwidth=17cm
\oddsidemargin=-0.5cm
\itemsep=-5pt

\usepackage{color}
\definecolor{webgreen}{rgb}{0,.5,0}
\definecolor{webbrown}{rgb}{.6,0,0}
\definecolor{webyellow}{rgb}{0.98,0.92,0.73}
\definecolor{webgray}{rgb}{.753,.753,.753}
\definecolor{webred}{rgb}{.85,0,0}
\definecolor{webblue}{rgb}{0,0,.8}

\usepackage[pdftex, bookmarksnumbered=true, breaklinks=true, 
  colorlinks=true, pdfstartview=FitV, linkcolor=webred, citecolor=webred,
  urlcolor=webblue]{hyperref}

% Astro definitions
\def\ang{\AA} 
\def\degpoint{\mbox{$^\circ\mskip-7.0mu.\,$}}
\def\halpha{\mbox{H$\alpha$}}
\def\hbeta{\mbox{H$\beta$}}
\def\hgamma{\mbox{H$\gamma$}}
\def\kms{\,km~s$^{-1}$}      % note leading thinspace
\def\lya{\mbox{Ly$\alpha$}}
\def\lyb{\mbox{Ly$\beta$}}
\def\minpoint{\mbox{$'\mskip-4.7mu.\mskip0.8mu$}}
\def\mv{\mbox{$m_{_V}$}}
\def\Mv{\mbox{$M_{_V}$}}
\def\peryr{\mbox{$\>\rm yr^{-1}$}}
\def\secpoint{\mbox{$''\mskip-7.6mu.\,$}}
\def\sqdeg{\mbox{${\rm deg}^2$}}
\def\squig{\sim\!\!}
\def\subsun{\mbox{$_{\normalsize\odot}$}}
\def\deg{\hbox{$^\circ$}}
\def\sun{\hbox{$\odot$}}
\def\earth{\hbox{$\oplus$}}
\def\lesssim{\mathrel{\hbox{\rlap{\hbox{\lower4pt\hbox{$\sim$}}}\hbox{$<$}}}}
\def\gtrsim{\mathrel{\hbox{\rlap{\hbox{\lower4pt\hbox{$\sim$}}}\hbox{$>$}}}}
\def\la{\mathrel{\hbox{\rlap{\hbox{\lower4pt\hbox{$\sim$}}}\hbox{$<$}}}}
\def\ga{\mathrel{\hbox{\rlap{\hbox{\lower4pt\hbox{$\sim$}}}\hbox{$>$}}}}
\def\sq{\hbox{\rlap{$\sqcap$}$\sqcup$}}
\def\arcmin{\hbox{$^\prime$}}
\def\arcsec{\hbox{$^{\prime\prime}$}}
\def\fd{\hbox{$.\!\!^{\rm d}$}}
\def\fh{\hbox{$.\!\!^{\rm h}$}}
\def\fm{\hbox{$.\!\!^{\rm m}$}}
\def\fs{\hbox{$.\!\!^{\rm s}$}}
\def\fdg{\hbox{$.\!\!^\circ$}}
\def\farcm{\hbox{$.\mkern-4mu^\prime$}}
\def\farcs{\hbox{$.\!\!^{\prime\prime}$}}
\def\fp{\hbox{$.\!\!^{\scriptscriptstyle\rm p}$}}
\def\micron{\hbox{$\mu$m}}
\def\case#1#2{\hbox{$\frac{#1}{#2}$}}
\def\slantfrac#1#2{\hbox{$\,^#1\!/_#2$}}
\def\onehalf{\slantfrac{1}{2}}
\def\onethird{\slantfrac{1}{3}}
\def\twothirds{\slantfrac{2}{3}}
\def\onequarter{\slantfrac{1}{4}}
\def\threequarters{\slantfrac{3}{4}}
\def\ubvr{\hbox{$U\!BV\!R$}}            % UBVR system
\def\ub{\hbox{$U\!-\!B$}}               % U-B
\def\bv{\hbox{$B\!-\!V$}}               % B-V
\def\vr{\hbox{$V\!-\!R$}}               % V-R
\def\ur{\hbox{$U\!-\!R$}}               % U-R
\def\jhk{\hbox{$J\!H\!K$}}              % JHK system
\def\jh{\hbox{$J\!-\!H$}}               % J-H
\def\hk{\hbox{$H\!-\!K$}}               % H-K
\def\jk{\hbox{$J\!-\!K$}}               % J-K
\def\nodata{\multicolumn{1}{c}{$\cdots$}}
\makeatletter
\def\ion#1#2{#1$\;${\small\rm\@Roman{#2}}\relax}
\makeatother

\begin{document}

\centerline{{\Large{\bf ESOFORM PDF\LaTeX\ MACROS}}}
\bigskip
\centerline{{\Large{\bf USER MANUAL FOR PHASE~1 PROPOSALS }}}
\bigskip
\centerline{{\bf European Southern Observatory }}
\vspace{3cm}

\def\period{100}

\section*{CONTENTS} 

\begin{enumerate}  
\item Introduction 
\item New features for Period 100
\item How to fill a Normal Programme template 
\item How to fill a Large  Programme template 
\item Submission of the application 
\end{enumerate} 

\vspace{3.0truecm}

\centerline{\large ESO DEADLINE FOR PROPOSAL SUBMISSION }
\medskip
\centerline{The ESO deadline for Period 100 proposal submission is:}
\medskip
\centerline{\huge \bf 30 March 2017, 12:00 CEST}

\vfill

\noindent This User Manual and the whole ESOFORM Package is
maintained by the Observing Programmes Office (OPO),
while the background software
is provided by the User Support System (USS) Department.

\break
%%%%%%%%%%%%%%%%%%%%%%%%%%%%%%%%%%%%%%%%%%%%%%%%%%%%%%%%%%%%%%%%%%%%%%

\section{INTRODUCTION}

The ESOFORM package has been designed to enable a fully
electronic preparation and submission of applications for observing
time on all telescopes located at the European Southern Observatory's
La Silla Paranal observatory.

{\bf Getting help.} Should you need assistance from ESO to prepare
your proposal, please send emails to the address
\href{mailto:esoform@eso.org}{\tt esoform@eso.org} for questions
related to the ESOFORM package as well as for more general questions
about Observatory policy, etc...
You should receive automatic acknowledgment from the system within 
4 hours. If you do not receive any acknowledgment of your query, 
please email 
\href{mailto:opo@eso.org}{\tt opo@eso.org} directly.

For instrument specific questions,
please email \href{mailto:usd-help@eso.org}{\tt usd-help@eso.org}.

%%%%%%%%%%%%%%%%%%%%%%%%%%%%%%%%%%%%%%%%%%%%%%%%%%%%%%%%%%%%%%%%%%%%%%

\subsection{How to Obtain the {\bf New ESOFORM} Proposal Package}

The ESOFORM Proposal Package may be obtained by logging into the ESO
User Portal at the following address:

\begin{center}
  \href{http://www.eso.org/UserPortal}{\bf
    \underline{http://www.eso.org/UserPortal}}. 
\end{center}

%%%%%%%%%%%%%%%%%%%%%%%%%%%%%%%%%%%%%%%%%%%%%%%%%%%%%%%%%%%%%%%%%%%%%%

\subsection{Description of the Content of the ESOFORM Proposal
  Package}

\noindent The ESOFORM package consists of:
\begin{itemize}  
\item two \LaTeX\ class files ({\tt esoform.cls} and {\tt
    esoformlarge.cls}) that, together with the style files ({\tt
    common2e.sty}
 and {\tt config.sty}), define
  all the macros required to generate the application forms for
  Normal and Large Programmes, respectively;
\item two template proposals ({\tt  template.tex} for Normal
  Programme applications, and  {\tt templatelarge.tex}  for  Large  Programme 
    applications), which the  users   may  edit directly in  order  to 
    create a new proposal;
\item this Users' Manual ({\tt usersmanual.tex}), which contains all the 
    information  required to fill the templates, as well as
    instructions on the electronic submission of proposals (via the
    Web-based WASP interface in the ESO User Portal);

\item a short README file.
\end{itemize} 

%%%%%%%%%%%%%%%%%%%%%%%%%%%%%%%%%%%%%%%%%%%%%%%%%%%%%%%%%%%%%%%%%%%%%%

\subsection{General Features}

The present manual describes the use of the ESOFORM templates,
which are composed of macros that are defined in the ESOFORM class and
style files. The macros allow the 
computer controlled typesetting of applications for observing time at
ESO.  If you are already familiar with \TeX\ or \LaTeX, you will
probably have no difficulty using the macros provided.  You should
follow the instructions given below and keep in mind that all your
input must conform to the standard \LaTeX\ rules.

The ESOFORM package has been prepared with the following
version of pdf\LaTeX: pdf\TeX, Version 3.141592 (Web2C 7.5.5).  If you
encounter any serious pdf\TeX\ or pdf\LaTeX\ problem, please send an
email to the address \href{mailto:esoformeso.org}{\tt esoform@eso.org},
describing the problem and indicating which version of pdf\LaTeX\ you
are using.  For ease of use, we have adopted (and already included in
the class files {\tt esoform.cls},  and {\tt esoformlarge.cls}) 
a number of \LaTeX\ definitions of commonly used
astronomical symbols (see list in Table~\ref{tab:symbols}).

\begin{table}[t]
\caption{Astronomical \LaTeX\ Symbols}
\label{tab:symbols}
\medskip
\begin{tabular*}{\hsize}{@{\extracolsep{0pt}}r@{\extracolsep{20pt}}l@{\extracolsep{\fill}}r@{\extracolsep{20pt}}ll@{\extracolsep{0pt}}}
\hline
\hline
& & & &\\
\verb"\ang"      & \ang      & \verb"90\deg"        & 90\deg        & \\
\verb"\halpha"   & \halpha   & \verb"16\sqdeg"      & 16\sqdeg      & \\
\verb"\hbeta"    & \hbeta    & \verb"28\arcmin"     & 28\arcmin     & \\
\verb"\hgamma"   & \hgamma   & \verb"11\arcsec"     & 11\arcsec     & \\
\verb"\lya"      & \lya      & \verb"5\fd4"         & 5\fd4         & \\
\verb"\lyb"      & \lyb      & \verb"8\fh2"         & 8\fh2         & \\
\verb"\mv"       & \mv       & \verb"2\fm56"        & 2\fm56        & \\
\verb"\Mv"       & \Mv       & \verb"10\fs08"       & 10\fs08       & \\
\verb"\ubvr"     & \ubvr     & \verb"23\fdg12"      & 23\fdg12      & \\
\verb"\ub"       & \ub       & \verb"3\farcm6"      & 3\farcm6      & \\
\verb"\bv"       & \bv       & \verb"0\farcs27"     & 0\farcs27     & \\
\verb"\vr"       & \vr       & \verb"0\fp4"         & 0\fp4         & \\
\verb"\ur"       & \ur       & \verb"\onehalf"      & \onehalf      & \\
\verb"\jhk"      & \jhk      & \verb"\onethird"     & \onethird     & \\
\verb"\jh"       & \jh       & \verb"\twothirds"    & \twothirds    & \\
\verb"\hk"       & \hk       & \verb"\onequarter"   & \onequarter   & \\
\verb"\jk"       & \jk       & \verb"\threequarters"& \threequarters& \\
\verb"\ion{C}{4}"& \ion{C}{4}& \verb"\slantfrac{{22}}{7}" & \slantfrac{{22}}{7} & (braces unless one character) \\
\verb"3.6\micron"& 3.6\micron& \verb"$\squig$" & $\squig$ & (math mode only) \\
\verb"25\kms"    & 25\kms    & \verb"$\lesssim$" & $\lesssim$ & (math mode only) \\
\verb"\peryr"    & \peryr    & \verb"$\gtrsim$"  & $\gtrsim$ & (math mode only) \\
\verb"M\subsun"  & M\subsun  & \verb"$\la$"      & $\la$ & (math mode only) \\
\verb"\sun"      & \sun      & \verb"$\ga$"      & $\ga$ & (math mode only) \\
\verb"\earth"    & \earth    & \verb"\nodata"  & \nodata & (tables only) \\
\verb"\sq"       & \sq       & \verb"" & & \\[4pt]
\hline
\end{tabular*}
\end{table}

For every observing period, the layout of the instruments will be
updated according to the anticipated availability of instruments at
the La Silla Paranal Observatory. The ESO Call for Proposals for the 
corresponding period has all the latest information on instrument availability
and any significant changes in performance. {\it Please note that {\bf only}
  proposals prepared using the {\bf latest} version of
  ESOFORM will be valid and accepted by ESO.}

\begin{itemize}

\item{\bf ESO proposal submission deadline: }
Please note that the ESO deadline will be strictly enforced: users should plan accordingly.
It is the PI's responsibility to resolve any problems related to LaTeX, figure uploads or configuration problems and verify the LaTeX proposal form well before the deadline as ESO cannot provide support beyond 11:00 CET on the day of the deadline.
The online receiver will switch off at 12:00 CET and no submissions or amendments to submitted proposals can be accepted after this time.
\end{itemize}

\newpage
%%%%%%%%%%%%%%%%%%%%%%%%%%%%%%%%%%%%%%%%%%%%%%%%%%%%%%%%%%%%%%%%%%%%%%

\section{IMPORTANT REMINDERS AND NEW FEATURES FOR PERIOD 100}
\label{sec:new}

\begin{itemize}

\item{\bf Normal and Large Programme forms:}
The Large Programme template ({\tt templatelarge.tex}) must be used to
prepare Large Programme proposals. The total amount of time requested in a programme determines which proposal template should be used.
Any programme requesting a total amount of time of 100 hours or more
must use the Large Programme template.
The Normal Programme template form {\bf must} be used for all other programmes,
including the Normal, ToO, Monitoring and Calibration proposal types, and GTO proposals
requesting under 100 hours.

\item{\bf Large GTO Programmes:} 
GTO teams can apply for up to four periods provided that this is compatible with the corresponding GTO agreement.
PIs should note that any GTO proposal requesting through the Large Programme channel is subject to the same requirements to provide a detailed delivery plan for data products and other conditions governing the reporting on the progress of Large Programmes.
If the GTO programme time request is for under 100 hours and only requires time in Period 100 the GTO team should fill ina Normal programme template form specifying the GTO programme type and the appropriate GTO contract keyword as usual. 
Should the programme require two periods the PI should fill in a Large Programme template,
keep the programme type as LARGE and enter the GTO contract keyword where indicated.
All GTO proposals will be evaluated and ranked together with Normal and Large Programme proposals in order to provide feedback to the GTO teams on the scientific standing of their GTO programme.


\item{\bf Large Programme Data Products: }
A key requirement of Large Programmes is the delivery of data products into the ESO Archive. PIs of Large Programmes are required to provide all data products within a reasonable timeframe. Further details are available in the Large Programmes Section of the Call for Proposals document.
A detailed delivery plan along with the relevant experience of proposing teams
should be outlined in Sections 6 and 7 of the Large Programme proposal template
(also see Section~\ref{sec:large}).

\item{\bf Monitoring programme proposals:}  
Proposers can apply for Monitoring Programmes to monitor targets over several periods.  
Monitoring Programme proposals should be prepared using the normal template form \verb|template.tex|. 
The Monitoring Programme type should be specified in this form as follows: \verb|\ProgrammeType{MONITORING}|, also see Table~\ref{tab:insmonitoring}.
More details on the definition of Monitoring Programmes and instruments available for this type of programme are given in Section~4.2 of the Call for Proposals.

\item{\bf Definition of Service Mode and Visitor Mode runs: }
%Remove in P95
An observing programme, as described in a single proposal, may consist of one or more runs.
Multiple runs should only be requested for observations with different instruments and/or for different observing modes (e.g., service mode, visitor mode or pre-imaging runs).  Proposers should split Visitor Mode observations at different epochs (e.g., due to different target RAs) into separate runs. Conversely, Service Mode runs should not be split according to time-critical windows, or used to group targets according to their Right Ascensions.


\item{\bf Observing conditions:} The definitions of the observing conditions for Phase 1 and Phase 2
can be found on the  \href{http://www.eso.org/sci/observing/phase2/ObsConditions.html}{\bf \underline{Observing Conditions}} webpage.
Please note that the seeing constraint to be specified should be 
{\bf the maximum acceptable seeing value (FWHM in arcseconds) in the V-band at zenith}.
Please see the Call for Proposals and the 
\href{http://www.eso.org/sci/observing/phase2/ObsConditions.html}{\bf\underline{Phase 2 Observing Conditions page}}
for more details.


\item{\bf Scheduling requirements in visitor mode: }
Any time constraints related to scheduling visitor mode runs should be flagged
using the \verb|\HasTimingConstraints| macro.
Any time constraints specified using the \verb|\TimeCritical| macro 
will be subject to a strict verification  at proposal submission. Please take
particular care to ensure
that the start and end dates of suitable observation windows comply
with the applicable time specification convention, and that the
amount of requested time (in the \verb|\ObservingRun| macro) fits
within the length of the slots indicated in the
\verb|\TimeCritical| macro.  \\
Several time intervals might be acceptable for each run 
(e.g., transits of extrasolar planets
across their host star); if this is the case the \verb|\TimeCritical| macro should be 
repeated as many   times as there are adequate alternative observation dates.
Further details and examples of the
correct usage are shown in Section~\ref{sec:timecrit}. 

\item \textbf{Precipitable water vapour (PWV):} Users of
%add VISIR & CRIRES here when they are available again
APEX instruments in Service Mode must specify an upper limit for PWV
  as an observing constraint during their Phase 1 and Phase 2
  preparation. Examples of the correct usage are shown in the ESOFORM package 
  template files.

 \item{\bf VLTI-ATs:} The baselines offered in Period 100 are listed on the
\href{http://www.eso.org/sci/facilities/paranal/telescopes/vlti/configuration/index.html}
{\bf\underline{VLTI baseline}} page.

\end{itemize}

%%%%%%%%%%%%%%%%%%%%%%%%%%%%%%%%%%%%%%%%%%%%%%%%%%%%%%%%%%%%%%%%%%%%%%

\section{HOW TO FILL A NORMAL PROGRAMME TEMPLATE}
\label{sec:normal}

As mentioned in the Introduction, you should fill in 
the template file ({\tt template.tex}) with your favourite editor. 
Instructions for Large Programmes
({\tt templatelarge.tex}) are given in Sect.~\ref{sec:large}. 
The easiest way to write a proposal is to modify the
template file {\tt template.tex} by following the examples therein and the
detailed instructions given in the present manual. Input in the
template is allowed {\bf only within the arguments of the provided
  ESOFORM macros}. The presence of text {\bf outside} the macro
arguments will lead to {\bf rejection} of the proposal by the
automatic proposal reception system (see Sect.~\ref{sec:submission}). 

Please note that {\bf it is the responsibility of the 
  applicants to stay 
  within the current box limits} and to eliminate potential
  overfill/overwrite problems. A careful visual check of the
  generated pdf file is mandatory.


\subsection{The Cycle, the Title, the Subcategory Code, the GTO,
  ToO, RRM, and XMM Flags: {\bf BOX~1}}

The first macros to check in the {\tt template.tex} files are as follows:
\begin{itemize}
\item \verb|\Cycle| contains the Period ID for this Call for
  Proposals, and should NOT be modified by the users.
\item \verb|\Title| must contain the title of the application (up to
  two lines).
\item \verb|\SubCategoryCode| must contain only one subcategory code,
  corresponding to the keyword (see Table~\ref{tab:categories}) best
  summarizing the aim of your proposal. 
  For example, a study of
  high-redshift clusters of galaxies will have the code A5. 
  Please note: CALIBRATION proposals MUST have a \verb|SubCategoryCode| of L0.
\item \verb|\ProgrammeType| should be {\tt NORMAL} for Normal Programmes, 
 {\tt CALIBRATION} for Calibration, {\tt GTO} for Guaranteed Time Observation 
 (GTO), {\tt MONITORING} for Monitoring Programmes, or {\tt TOO} for Target of 
 Opportunity (ToO) programmes.  
 {\tt LARGE} types are also defined for Large Programmes, but they can only be 
 used within the corresponding specific template, {\tt templatelarge.tex} 
 (see Sect.~\ref{sec:large}). The programme type, {\tt LARGE}, should also be used 
 for GTO teams submitting a GTO proposal that runs over two periods (see the Call for Proposals
for more details). \\
  A CALIBRATION proposal MUST have a SubCategoryCode of L0.\\
  Note that a GTO proposal should ask only for GTO time; it is
  compulsory to fill another, non-GTO proposal if you need more non-GTO
  time, even if it is for exactly the same project. If you submit a
  TOO proposal, your proposal must include a completed ToO page
  (see Sect.~\ref{sec:toopage}). If you submit a GTO proposal, you
  must additionally uncomment and fill in the macro
  \verb|\GTOcontract|, which is described below.
\item \verb|\GTOcontract| {\em must\/} be uncommented if the proposal
  type is GTO. This macro takes one mandatory argument: 
  the keyword corresponding to the contract or agreement
  governing the allocation of Guaranteed Time under which the present
  proposal is submitted. The applicable keyword for each contract or
  agreement has been communicated by ESO to the coordinator of the
  respective GTO Team or to the designated contact person. Every
  single GTO proposal is allowed to request time within the framework
  of only one GTO contract or agreement. If your project involves
  Guaranteed Time corresponding to several different agreements, you
  must submit separate proposals for the time to be charged to each
  agreement. 
\item \verb|\ObservationInRRM| must be uncommented if your proposal is
  a Target of Opportunity proposal applying for Rapid Response Mode
  (RRM) observations.
\end{itemize}

\ifodd\period
 \begin{itemize}
  \item \verb|\ObservationWithXMM| must be uncommented if your proposal
   is applying for time under the VLT-XMM agreement (see the Call for
   Proposals for details on this agreement).  Note that in this case
   users can apply for both Periods 100 and 101.
 \end{itemize}
\else
 \begin{itemize}
 \item \verb|\ObservationWithXMM| should be left commented, as for
   Period 100 it is not possible to apply for time under the
   VLT-XMM agreement.
 \end{itemize}
\fi

%%%%%%%%%%%%%%%%%%%%%%%%%%%%%%%%%%%%%%%%%%%%%%%%%%%%%%%%%%%%%%%%%%%%%%

\begin{table}[p]
\caption{ESO OPC categories and subcategories}
\label{tab:categories}
\medskip
{\small\sf
\begin{center}
\begin{tabular}{llcl} 
\hline\hline
& & & \\
Panels & Categories & Code & Subcategories \\
& & & \\
\hline 
& & & \\[-6pt]
{\bf A}& Cosmology & A1 & Surveys of high-z galaxies; \\
       &           & A2 & Identification studies of extragalactic surveys; \\
       &           & A3 & Large scale structure and evolution; \\
       &           & A4 & Distance scale; \\  
       &           & A5 & Groups and clusters of galaxies; \\  
       &           & A6 & Gravitational lensing; \\  
       &           & A7 & Intervening absorption line systems; \\  
       &           & A8 & High-redshift galaxies (star formation and ISM); \\
       &           & A9 & Surveys of AGNs and AGN host galaxies. \\[4pt]
%& & & \\
\hline
& & & \\[-6pt]
{\bf B}& Galaxies  & B1 & Morphology and galactic structure; \\  
       & and       & B2 & Unresolved and resolved stellar populations; \\  
       & galactic nuclei & B3 & Chemical evolution; \\  
       &           & B4 & Galaxy dynamics; \\  
       &           & B5 & Peculiar/interacting galaxies; \\   
       &           & B6 & Non-thermal processes in galactic nuclei (incl. \\
       &           &    &  QSRs, QSOs, blazars, Seyfert galaxies, BALs, \\ 
       &           &    &  radio galaxies, and LINERS); \\  
       &           & B7 & Thermal processes in galactic nuclei and starburst\\
       &           &    &  galaxies (incl. ultraluminous \\  
       &           &    &  IR galaxies, outflows, emission lines, and \\
       &           &    &  spectral energy distributions); \\  
       &           & B8 & Central supermassive objects. \\[4pt]
%& & & \\
\hline
& & & \\[-6pt]
{\bf C}& ISM, & C1 & Gas and dust, giant molecular clouds, cool and hot gas,\\
       & star formation &   &  diffuse and translucent clouds; \\ 
       & and            &C2& Chemical processes in the interstellar medium; \\
       & planetary systems &C3& Star forming regions, globules, protostars, \\
       &      &    &  HII regions; \\
       &      & C4 & Pre-main-sequence stars (massive PMS stars, \\
       &      &    &  Herbig Ae/Be stars and T Tauri stars); \\  
       &      & C5 & Outflows, stellar jets, HH objects; \\
       &      & C6 & Main-sequence stars with circumstellar matter, \\
       &      &    &  early evolution;   \\
       &      & C7 & Young binaries, brown dwarfs, exosolar planet searches;\\ 
       &      & C8 & Solar system (planets, comets, small bodies). \\[4pt]  
%& & & \\
\hline
& & & \\[-6pt]
{\bf D}& Stellar   & D1 & Main-sequence stars; \\
       & evolution & D2 & Post-main-sequence stars, giants, supergiants, \\
       &           &    &  AGB stars, post-AGB stars; \\
       &           & D3 & Pulsating stars and stellar activity; \\
       &           & D4 & Mass loss and winds; \\
       &           & D5 & Supernovae, pulsars; \\
       &           & D6 & Planetary nebulae, nova remnants and \\
       &           &    &  supernova remnants; \\
       &           & D7 & Pre-white dwarfs and white dwarfs, neutron stars; \\ 
       &           & D8 & Evolved binaries, black-hole candidates, novae, \\ 
       &           &    &  X-ray binaries, CVs; \\
       &           & D9 & Gamma-ray and X-ray bursters; \\
       &           & D10& OB associations, open and globular clusters, \\
       &           &    &  extragalactic star clusters; \\
       &           & D11& Individual stars in external galaxies, resolved stellar populations;\\
       &           & D12& Distance scale -- stars.\\[4pt]
%& & & \\
\hline
%& & & \\
\end{tabular}
\end{center}
}
\end{table}

Your first sequence will then have the following general format:
\begin{verbatim}
        \Cycle{100A}
        \Title{AGN host galaxies}
        \SubCategoryCode{B9}
        \ProgrammeType{GTO}
        \GTOcontract{<your-contract-keyword>)
\end{verbatim}
\noindent which means that you would like to study some AGN host
galaxies, with subcategory code B9, and this would be Guaranteed Time
Observations within the framework of the contract or agreement
referred to by keyword \verb|<your-contract-keyword>|.

%%%%%%%%%%%%%%%%%%%%%%%%%%%%%%%%%%%%%%%%%%%%%%%%%%%%%%%%%%%%%%%%%%%%%%

\subsection{The Abstract/Total Time Requested:  {\bf BOX 2}}

The ESOFORM includes a string called the
``Total Time Requested'' in Box 2. 
This field remains blank when compiling locally; this 
is normal and proposers should not be concerned about it.
The total time requested is only computed in the version of the proposal 
that is reviewed by the OPC. 

This macro (\verb|\Abstract|) contains the  abstract of the  proposal,
i.e., a brief summary, in up to nine lines, of your scientific aim.
\begin{verbatim}
          \Abstract{ .
                     .
                     .
                     The text of your summary which will usually be 
                     several lines long. Line breaking will  
                     automatically be taken care of by LaTeX.
                     .
                     .
                     .              }  <-- Do not forget the
                                           closing brace !
\end{verbatim}



%%%%%%%%%%%%%%%%%%%%%%%%%%%%%%%%%%%%%%%%%%%%%%%%%%%%%%%%%%%%%%%%%%%%%%

\subsection{Information about the Different Runs: {\bf BOX 3}}
\label{sec:obsrun}

The next macro (\verb|\ObservingRun|) allows the description of the
different parameters characterising your observing run(s) and is
necessary for the scheduling and completion of your programme (see
examples below). This macro takes ten arguments.
These must be specified between the ten pairs of curly braces \verb|{}|, 
which are related to the parameters described below.

\medskip

{\bf 1. RUN ID.} Your programme may involve several observing
runs, e.g. for complementary use of different telescopes or
different instruments. Each observing run (up to 26) required by a
proposal should be identified by a different letter, following the
sequence A, B, C, \dots, Z as needed.  Provide, in the first pair of
curly braces, the run identification(s).  For example,
\begin{verbatim}
    \ObservingRun{A}{}{}{}{}{}{}{}{}{}
    \ObservingRun{B}{}{}{}{}{}{}{}{}{}
\end{verbatim}
A Normal Programme may have up to 26 runs. Since the space for the
run description in Box 3 is limited to 10 lines, a new box containing
the observing runs beyond this limit will be created at the end of the
proposal form if needed. 

\medskip

{\bf 2. PERIOD KEYWORD.} Provide the period number in the second pair
of curly braces.  For normal proposals this must always be 100,
while for  Monitoring programmes this can span up to 4 periods.

\ifodd\period
 Uncomment for odd periods
 However, if users are applying under the VLT-XMM agreement,
 a period of 101 is also valid.
\fi

\medskip

{\bf 3. INSTRUMENT.} Provide the keyword of the instrument required
for each observing run. The complete list of keywords of all
instruments offered in Period 100 for Normal Programmes is
given in Table~\ref{tab:insnormal}. 

\begin{table}[h]
\caption{Keywords of Available Instruments for Normal Programmes}
\label{tab:insnormal}
\medskip
\begin{center}
\begin{tabular}{@{\extracolsep{0pt}}l@{\extracolsep{40pt}}l@{\extracolsep{0pt}}}
\hline
\hline \\[-6pt]
Telescope&Instrument keywords\\[4pt]
\hline \\[-6pt]
UT1  &FORS2, KMOS, NACO\\
UT2  &FLAMES, UVES, XSHOOTER\\
UT3  &SPHERE, VIMOS$^1$, VISIR\\
UT4$^2$  &HAWKI, MUSE, SINFONI\\
VLTI &AMBER, GRAVITY, PIONIER\\
VISTA&VIRCAM\\
VST  &OMEGACAM$^3$\\
NTT  &SOFI, EFOSC2, ULTRACAM, SpecialNTT\\
3.6  &HARPS, Special3.6\\
APEX & ARTEMIS, LABOCA, \\
&    FLASH$^4$,SEPIA$^5$, SHFI, SpecialAPEX\\
\hline
\end{tabular}
\end{center}
$^1$ VIMOS is only available in Service Mode for filler programmes requiring bad seeing and/or transparency;  VIMOS runs will not be carried over.\\
$^2$ Programmes requesting time on  UT4 should take into account the provisional timeline and UT4 activities described in the Call for Proposals.\\ 
$^3$ OMEGACAM is only available for programmes that request poor weather conditions (see the Call for Proposals for details). However, TOO proposals will also be considered. \\
$^4$ FLASH is an APEX PI instrument; in order to propose for its use the instrument PI must be contacted at least two weeks prior to submitting the proposal (see the Call for Proposals).\\
$^5$ Both the band 5 and band 9 receivers are offered.\\
\end{table}

\begin{table}[h]
\caption{Keywords of Available Instruments for Monitoring  Programmes}
\label{tab:insmonitoring}
\medskip
\begin{center}
\begin{tabular}{@{\extracolsep{0pt}}l@{\extracolsep{40pt}}l@{\extracolsep{0pt}}}
\hline
\hline \\[-6pt]
Telescope&Instrument keywords\\[4pt]
\hline \\[-6pt]
UT1       & FORS2, KMOS\\
UT2       & FLAMES, UVES, XSHOOTER \\
UT3       & SPHERE\\%VISIR$^1$, VIMOS
UT4$^1$   & HAWK-I, MUSE\\
VLTI$^2$  & GRAVITY, PIONIER\\
APEX$^3$  & ARTEMIS, SEPIA$^4$ \\
\hline
\end{tabular}
\end{center}
$^1$ Monitoring Programmes for UT4 instruments must be compatible with
constraints imposed by the UT4 activities in the Call for Proposals, 
taking into account that the schedule described is provisional. \\
$^2$ Monitoring Programmes on the VLTI must be compatible with the
VLTI and (if relevant) UT4 activities described in the Call for Proposals,
 taking into account that the schedule described is provisional.\\
$^3$ APEX observations for approved MPs can only be carried out in
the ESO time-slots. \\
$^4$ Both the band 5 and band 9 receivers are offered.\\
\end{table}



Provide, in the third pair of curly braces, the instrument(s) required
for each observing run.  

For example, 
\begin{verbatim}
    \ObservingRun{A}{100}{FORS2}{}{}{}{}{}{}{}
\end{verbatim}

For visitors bringing their own Visitor Instrument, please use
the corresponding ``Special'' keyword, e.g., ``Special3.6'' for 
an instrument to be mounted on the 3.6, and ``SpecialNTT'' for the NTT.
It is then compulsory to fill in the Visitor Instrument information
page (see Sect.~\ref{sec:visins}).


\medskip

{\bf 4.  REQUESTED TIME.}  In order to allow for the automated
scheduling of proposals, you must specify the amount of time 
that you are requesting (hours in Service Mode and nights in Visitor
Mode). 

\smallskip

{\bf For Service Mode (SM)}, at this stage only provide the total
number of hours requested, followed by the letter \verb|h| for hours.
This should include also the time related to any special calibrations
required in addition to the standard calibrations provided by ESO.
Any more detailed information about possible particular scheduling
features will be provided during the preparation of the Phase~2
Service Mode proposal.

\smallskip

{\bf For Visitor Mode (VM) proposals}, the runs are scheduled
according to the information supplied in the Phase~1 proposal
form.  Consequently, you are required to provide detailed information 
about any particular scheduling requirements for the successful 
completion of your programme in this form.  You
should code the requested time (in nights), starting with the total
time requested, followed by the letter \verb|n| for nights, according
to the examples below.
If runs are to be split they should be specified using the following
convention:

%\begin{center}
\begin{tabular}{ll}
& \\
4n          & for 4 consecutive nights, \\
4n=4x1      & for 4 times 1 night, with intervals in between,\\
6n=3x2      & for 3 times 2 consecutive nights, \\
7n=3x2+1 \,\,\,\, & for 3 times 2 consecutive nights followed by 1 night. \\
\end{tabular}
%\end{center}

\bigskip

If the programme needs run durations of a half-night, you can code the
requested time (in nights and half-nights), starting with the total
time requested, following the few examples below (with {\bf H1} =
first half of the night, and {\bf H2} = second half of the night):

%\begin{center}
\begin{tabular}{ll}
& \\
2n=4H1     & for 4 consecutive first halves of a night, \\
2n=4x1H2   & for 4 times the second half of a night, with intervals in between,\\   
3n=3x2H1  & for 3 times 2 consecutive first halves of a night, \\   
8n=3x2+4H2& for 3 times 2 consecutive nights followed by 4 consecutive second halves of a night, \\
8n=3x2+4x1H1& for 3 times 2 consecutive nights followed by 4 non-consecutive first halves of a night. \\
\end{tabular}
%\end{center} 

\bigskip

There is flexibility in this macro so that sub-runs
can be specified for fractions of a night  (not just half-nights).
Where a decimal value precedes the H, H1 or H2 values, they specify the
fraction of the night that is required for each observation. Only
one decimal place is allowed.
For example:

\begin{tabular}{ll}
& \\
0.6n=2x0.6H1     & for 2 consecutive 0.3n runs -- starting in the first half of the night, \\
1.8n=3x1.2H2    & for  3 consecutive 0.6n runs -- starting at the end of the first half of the night, \\
2.5n=2H1+2x1.6H2 & for 2 consecutive first halves of a night followed by 2 x 0.8n runs.\\
\end{tabular}

Please note that designations such as 2.0n=2x2.0H1 are invalid. 2.0H1 implies a run that is one whole night and will therefore be rejected by the proposal receiver.
In this case the user should specify 2n=2x1.
Similarly any values greater than or equal to 2.0 preceding
H, H1 or H2 will be rejected.

\bigskip

Provide the total amount of time
that is required for the observing run in the fourth pair of curly braces, with details about
possible sub-runs.  For example,

\begin{verbatim}
    \ObservingRun{A}{100}{FORS2}{8n=3x2+4H2}{}{}{}{}{}{}
\end{verbatim}
{\bf For all non-consecutive schedules, the details of the time
  intervals between the different sub-runs must be provided in Box
  12} (see Sect.~\ref{sec:schedreq}). \\


\medskip

{\bf 5. MONTH PREFERENCE.} Provide the first three letters of the
month (e.g.\ dec) that is your first preference for
scheduling (valid months are the ones included in the current period,
namely oct, nov, dec, jan,
feb, mar).
%; all months are however valid for VLT-XMM proposals).  
For Service Mode runs ``any'' should be entered here with any specific 
requirements detailed in Box 12. 
For example,
\begin{verbatim}
    \ObservingRun{A}{100}{FORS2}{8n=3x2+4H2}{dec}{}{}{}{}{}
\end{verbatim}
Please note that this month preference should be consistent with the time
constraints specified in Box 12 (Sect.~\ref{sec:schedreq}).

\medskip

{\bf 6. MOON REQUIREMENT.} Provide the required phase of the moon, by
using only one of the following three characters (see the Call for
Proposals for the exact definition), namely:
\begin{itemize}
\item {\tt d} for ``dark time''
\item {\tt g} for ``grey time''
\item {\tt n} for ``no restriction''
\end{itemize}
For example,
\begin{verbatim}
    \ObservingRun{A}{100}{FORS2}{8n=3x2+4H2}{dec}{d}{}{}{}{}
\end{verbatim}

\medskip

{\bf 7. SEEING REQUIREMENT.} Provide the required maximum acceptable seeing value (FWHM in arcseconds) in the V-band at zenith (see the Call for Proposals). The value should correspond to the one given as input to the ETC.  Your requirement must be one of the following values:

\smallskip

0.4, 0.6, 0.8, 1.0, 1.2, 1.4, n

\smallskip

For example,
\begin{verbatim}
    \ObservingRun{A}{100}{FORS2}{8n=3x2+4H2}{dec}{d}{0.8}{}{}{}
\end{verbatim}

\medskip

{\bf 8. TRANSPARENCY REQUIREMENT.} Provide the transparency
condition of the atmosphere required during your observations (see the
Call for Proposals for the exact definition). Your requirement must
be one of the following values:

\smallskip

\begin{tabular}{ll}
  CLR & for clear sky, although with some rare clouds \\
  PHO & for photometric, a perfect night \\
  THN & for thin cirrus, inducing absorption up to 0.2 mag \\
\end{tabular}

\smallskip

For example, 
\begin{verbatim}
    \ObservingRun{A}{100}{FORS2}{8n=3x2+4H2}{dec}{d}{0.8}{PHO}{}{}
\end{verbatim}

\medskip

{\bf 9.  OBSERVING MODE.} Provide the requested observing mode:
\verb|v| = Visitor Mode and \verb|s| = Service Mode. For example,
\begin{verbatim}
    \ObservingRun{A}{100}{FORS2}{8n=3x2+4H2}{dec}{d}{0.8}{PHO}{v}{}
\end{verbatim}

{\bf 10.  RUN TYPE.}
This should only be filled in if the programme is a TOO or GTO 
and the run is a TOO Run.  The RUN TYPE field must be left blank 
for all Normal and Calibration Programme proposals.

For programme types that allow TOO Runs (TOO, GTO and DDT), this feature
should be used only for ToO runs.
Briefly, ToO Runs are defined to be runs for which the target cannot be known more than
one week before the observation must be carried out.
If you want to specify one of your runs 
as being a ToO run, then please enter ``TOO'' in the tenth (and final) brackets.

\begin{verbatim}
    \ObservingRun{A}{100}{FORS2}{1h}{dec}{d}{0.8}{PHO}{s}{TOO}
\end{verbatim}

More details corresponding to this ToO run must be specified in the
\verb|\TOORun| macro (Sect.~\ref{sec:toopage}).
This field should be left blank for all other (non-TOO) runs.



\subsubsection*{Alternative runs}
For each requested run, you may specify one or several
``alternative runs'' for possible execution of the proposed
observations with another instrument (in general mounted on another
telescope). To this effect, add another line in Box~3, with in the
first pair of curly braces, the letter identifying your primary run,
followed by ``\verb|/alt|''. For example,
\begin{verbatim}
    \ObservingRun{A}{100}{FORS2}{20h}{dec}{d}{0.8}{CLR}{s}{}
    \ObservingRun{A/alt}{100}{VIMOS}{6n}{dec}{d}{0.8}{CLR}{v}{}
\end{verbatim}
indicates that the observations of run A could be obtained
through allocation either of 20 hours in Service Mode with FORS2
(primary choice) or of 6 nights in Visitor Mode with VIMOS (secondary
choice). You may specify several alternative runs for each primary run
(e.g., in the example above, an EFOSC2 run might be a suitable alternative).

\subsubsection*{Multiple runs}
If more than one run is needed for execution of
the programme, then fill as many lines as needed.  For example,
\begin{verbatim} 
    \ObservingRun{A}{100}{FORS2}{40h}{dec}{d}{0.8}{PHO}{s}{}
    \ObservingRun{A/alt}{100}{VIMOS}{4n=2x4H1}{dec}{d}{0.8}{PHO}{v}{}
    \ObservingRun{B}{100}{NACO}{3n=3x1}{jan}{n}{0.6}{CLR}{v}{}
    \ObservingRun{C}{100}{FORS2}{2n=2x1}{jan}{n}{0.6}{CLR}{v}{}
\end{verbatim}

APEX users should note that all observations for a given APEX
instrument must be included in a {\bf single run} for each period. The proposal
receiver 
will reject any proposal with more than one run per period per APEX instrument. 

\subsubsection*{Programmes using both EFOSC-2 and SOFI at La Silla}
\label{sec:sofiefosc}

As described in the Call for Proposals,
proposals involving observations with both EFOSC-2 and SOFI,
for which the amount of time with either instrument is less than
3 nights can be carried out at La Silla.
Note that the runs should be contiguous and the requested time
summed over both the EFOSC-2 and SOFI runs is $\ge$ 3 nights.
These types of programmes should be specified
by using the pseudo-instrument SOFOSC in the Instrument field and
the total amount of time over the EFOSC-2 and SOFI runs.


For instance, if you require 3 nights on EFOSC-2 and 2 nights on SOFI, the
combined run would be specified as follows using the \verb|\ObservingRun|
 macro:
\begin{verbatim}
    \ObservingRun{A}{100}{SOFOSC}{5n}{dec}{d}{0.8}{PHO}{v}
\end{verbatim}

and in the justification for the requested amount of time
(Box 8, macro \verb|\WhyNights|), you should indicate the actual
lengths of each of the two runs. 
SOFOSC runs (i.e. combined EFOSC-2 and SOFI runs)  with a
total duration of less than 3 nights are not accepted.

Please specify the instrument configuration for the longest run
in the macro \verb|\INSconfig|. For example, if the longest run
is with the EFOSC-2 instrument:
\begin{verbatim}
   \INSconfig{A}{SOFOSC}{Spectro-long-slit}{Grism\#2:510-1100}
\end{verbatim}

\subsubsection*{Proprietary time} 
The default data proprietary time is 12 months.  Nevertheless, you can
ask to reduce it for 
your data by using the macro \verb|\ProprietaryTime{|{\it
    time\/}\verb|}|.  The {\it time\/} is expressed in months, and
only the following values can be entered:  0, 1, 2, 6, 12.  For
example, 
\begin{verbatim}
    \ProprietaryTime{6}
\end{verbatim}

Please note that this macro does not produce any printable output at
compilation, but the information that it contains will be duly stored
in ESO's database when the proposal is submitted.

\subsection{Past, Present, and Future of this Programme:  {\bf BOX 4}}

In order to allow for the evaluation of the proposal within the
broader context of the project of which it is part, taking into
account the observations already obtained in the past and the data
still to be acquired in the future, indicate in Box~4:
\begin{itemize}
\item \verb|\AwardedNights|: the amount of time (in nights or hours)
  allocated to this project in previous periods, together with the
  programme number (e.g., 098.B-1234), and the telescope
  on which this time was allocated;
\item \verb|\FutureNights|: the amount of time (in nights or hours)
  still necessary, in the future, after this proposal, to complete the
  programme, if any, and the corresponding telescope(s). 
\end{itemize}

For example,
\begin{verbatim}
    \AwardedNights{UT1}{4n in 098.B-1234}
    \FutureNights{UT3}{20h}
\end{verbatim} 

\subsection{Special Remarks:  {\bf BOX 5}}

Take advantage of this box to provide any special remark (up to three
lines).  For example,
\begin{verbatim}
  \SpecialRemarks{This programme is a resubmission, in updated form, of
    proposal 098.B-1234, which had been granted 2n in VM
    with UT2+UVES and was entirely clouded out.}
\end{verbatim}

\subsection{Name and Affiliations of PI and CoI(s): {\bf BOX 6}}

The macro \verb|\PI| must be used to identify the Principal
Investigator (PI) of the proposals. Its parameters are simply the 
ESO User Portal username of the PI. Do not fill in the PI's full
name as the receiver will convert the ESO User Portal username
to the PI's full name on submission.
Usage of this macro is illustrated in the following example, please note that the username is case-sensitive:
\begin{verbatim}
    \PI{JSMITH99}
\end{verbatim}

You should use the macro \verb|\CoI| to specify also, for
all the Co-Investigators (CoIs) of this proposal, their initial(s),
last name, and institute code. Institutes and their codes are listed 
according to country at the following webpage:\\
\href{http://www.eso.org/sci/observing/phase1/countryselect.html}
{\bf\underline{http://www.eso.org/sci/observing/phase1/countryselect.html}}.\\
You should have one instance of the macro \verb|\CoI| for each CoI of the
proposal. Even if several CoIs share the same institute
please use a separate macro for each CoI. The number of CoI macros you can 
use is in principle unlimited. However, please note that 
even if all CoIs do not appear in the printed version of the form,
the entire  CoI list is stored in the ESO database, where it
can be accessed when required.

Please note: Due to the way in which the proposal receiver system parses
the CoI macro, the number of pairs of curly brackets, \verb|{}|,
in this macro must be strictly equal to 3, i.e., the
number of parameters of the macro. Accordingly, curly
brackets should not be used within the parameters (e.g.,
to protect LaTeX signs).

\noindent
For instance:
\begin{verbatim}
   \CoI{L.}{Ma\c con}{1150}
   \CoI{R.}{Men\'endez}{1098}
\end{verbatim}

\noindent
are valid, while

\begin{verbatim}
   \CoI{L.}{Ma{\c}con}{1150}
   \CoI{R.}{Men{\'}endez}{1098}
\end{verbatim}

\noindent
are not. Unfortunately the receiver does not give an
explicit error message when such invalid forms are
used in the CoI macro, but the processing of the proposal
keeps hanging indefinitely.

\vspace{0.5cm}
An example of a CoI list follows:
\begin{verbatim}
   \CoI{H.}{Cerny}{1150}
   \CoI{S.}{Bailer-Brown}{1088}
   \CoI{K.L.}{Giorgi}{1164}
   \CoI{S.}{Lichtman}{2047}
   \CoI{L.}{Men\'endez}{1150}
\end{verbatim}

When the proposal is submitted each institute code will be converted to the
corresponding country and institute by the online receiver.


\subsection{Description of the Proposed Programme: {\bf BOX 7}}

The next two pages contain the description of the proposed programme,
including all text, references and figures.
This is restricted to TWO pages *including* figures. The
text sections are composed of two required components, which are 
activated by two macros.

A -- Scientific rationale: this section should describe the scientific
background of the project, with pertinent references; any previous
work in the field plus the justification for the present proposal
should be included.  The content of this section should be placed
between the curly braces of the macro \verb|\ScientificRationale{}|. 

B -- Immediate objective of the proposal: this section should state what
is actually going to be observed and what will be extracted from the
observations, so that the feasibility becomes clear.  The content of
this section should be placed between the curly braces of the
macro \verb|\ImmediateObjective{}|.
%VLT-XMM: If this is an odd period uncomment the following.
%If this is a VLT-XMM proposal, the immediate objectives to be met with the XMM observations should also be described here.

The references should preferably use the simplified abbreviations used 
in {\em Astronomy \& Astrophysics\/}.

\medskip

\noindent{\bf THE RELATIVE LENGTHS OF EACH OF THESE 
SECTIONS ARE VARIABLE. ALL TEXT AND FIGURES MUST FIT WITHIN A TOTAL
OF TWO PAGES. } Any text not fitting within the allocated page can be
added to the next page (with the figures). 
It is the responsibility of the proposers to check that their
programme description does not exceed the maximum acceptable
length. To this effect, proposers should carry out a careful visual
inspection of a print-out of their proposal prior to submitting
it. Also, when the proposal is compiled with pdf\LaTeX, the length of
the text is checked, and a warning message is issued if it is greater
than 2 pages. While this warning may easily be overlooked in the
real-time terminal window from which pdf\LaTeX\ is run because of the
continued scrolling resulting from other output, it is recorded in the
pdf\LaTeX\ logfile. 

%%%%%%%%%%%%%%%%%%%%%%%%%%%%%%%%%%%%%%%%%%%%%%%%%%%%%%%%%%%%%%%%%%%%%%

\subsection{Figures: {\bf BOX 7 (cont'd)}}

Up to ONE page of figures and text can be added in addition to the 
page detailing the Scientific Justification and the Immediate Objective.
There is a size limit of 1MB for each figure to be uploaded.
This material can be included using the 
macros \verb|\MakePicture{}{}| and \verb|\MakeCaption{}|. Any
additional text from the sections in the previous page (e.g.,
Immediate Objective) can also be included using the 
\verb|\MakeCaption{}|.

{\bf NOTE THAT POSTSCRIPT PICTURES ARE NOT ACCEPTED.} Since the
proposals are compiled using the pdf\LaTeX\ package, only JPEG and PDF
file formats are accepted.  Attachments in other formats should be
converted into one of the accepted formats using appropriate tools
(such as ps2pdf, convert, or gimp).  In order to reduce the size of
the attachments, {\bf we strongly suggest the use of the PDF format for
  simple plots and graphs, and JPEG for large figures (such as
  astronomical images).} 

The figure macro \verb|\MakePicture{}{}| has two arguments: the name
of the file of the picture, and a list of optional keywords specifying
formatting parameters of the image (as defined in the {\tt graphicx}
package). For example:
\begin{verbatim}
    \MakePicture{MyPic1.pdf}{width=15cm,height=8.0cm,angle=90}
    \MakePicture{MyPic2.jpg}{width=12cm}
\end{verbatim}
The filename should have a {\tt .jpg} or {\tt .jpeg} extension for
JPEG files, and a {\tt .pdf} extension for PDF files; other extensions
are not accepted. If there is no extension in the filename the receiver
will hang and submission will not be possible.

The  caption macro \verb|\MakeCaption{}| takes one single  argument,
which should contain any \LaTeX\ caption. For example:
\begin{verbatim}
    \MakeCaption{Insert caption using LaTeX.}
\end{verbatim}

These attachments will be printed on up to one page
immediately following the scientific description. You must check the
pdf output generated by pdf\LaTeX\ before submitting your proposal to
make sure that the attachments are properly included. In particular,
{\bf colour figures should still be readable if printed in  
black and white}. Also,
it is {\bf your responsibility} to check that your attachments 
{\bf fit within the allocated 1 page}. Please note that
when the proposal is compiled with pdf\LaTeX, the space required by
the attachments is checked, and a warning message is issued if it 
exceeds 1 page. While this warning may easily be overlooked in the
real-time terminal window from which \LaTeX is run because of the
continued scrolling resulting from other output, it is recorded in the
logfile generated by LaTeX. You are strongly encouraged to check
this log file.

\subsection{Justification of Requested Time: {\bf BOX 8}}

In this box, you should provide a careful justification of the
requested lunar phase and of the requested amount of time. 
To this effect, you should use the ESO Exposure
Time Calculators whenever possible; these exist for all Paranal and La Silla
instruments and are available at
\href{http://www.eso.org/observing/etc}{\bf
  \underline{http://www.eso.org/observing/etc}}. 

Links to exposure time calculators for APEX instrumentation
  can be found in Section 7 of the Call for Proposals.

For each telescope and instrument to be used, please specify the
version of the ESO Exposure Time Calculator that you have
used. Do {\bf not} include any correction for unexpected
meteorological conditions. The text should be typed as arguments of
the following two macros:
\begin{verbatim}
    \WhyLunarPhase{}
    \WhyNights{}
\end{verbatim} 


\subsection{Telescope Justification: {\bf BOX 8a}}
This section should provide a justification for the use of the selected telescope(s) (e.g., VLT, NTT, etc...) with respect to other available alternatives. The content of this section should be placed between the curly
braces of the macro \verb|\TelescopeJustification{}|.


\subsection{Observing Mode Justification: {\bf BOX 8b}}
This section should provide a justification for the observing mode requested (Visitor or Service). The
content of this section should be placed between the curly braces of the macro \verb|\ModeJustification{}|.

\subsection{Calibration Request: {\bf BOX 8c}}

For Service Mode runs, the calibrations foreseen in the instrument
calibration plans are 
absorbed by the Observatory; they do not need to be included in the
amount of requested time. In Visitor Mode, up to
30\,min per night is devoted to the acquisition of these calibrations
by the Observatory staff. If, in order to achieve the scientific goals
of your projects, calibrations not foreseen in the respective
calibration plan are required, you must include the additional amount
of time that is needed to obtain them in the total amount of time that
you are requesting. 

The macro \verb|\Calibrations| must be used to specify the calibration
requirements of your proposal. It takes two arguments. The first one
should be set to \verb|standard| if the calibrations contemplated in
the calibration plan are sufficient. In this case, no input is
required for the second argument:
\begin{verbatim}
    \Calibrations{standard}{}
\end{verbatim} 
If, on the other hand, you need
additional calibrations, the first argument must be set to
\verb|special|, and a brief description of non-standard calibrations
that you need must be given as second argument. For example,
\begin{verbatim}
    \Calibrations{special}{Adopt a special calibration}
\end{verbatim} 
Note that non-standard
daytime calibrations must be specified here, but contrary to
additional nighttime calibrations, the corresponding time need not 
be included in the total amount of requested time. 


\subsection{Last Use of ESO Facilities: {\bf BOX 9}}
The macro \verb|\LastObservationRemark| must be used to 
provide a brief report on the use of the ESO facilities during the
last 2 years. You should specify the programme identification
numbers, and describe the status of the data obtained, and the 
scientific output generated.

\subsection{ESO Archive: {\bf BOX 9a}}
You should use the \verb|\RequestedDataRemark| macro to indicate if
the data requested in the proposal are in the ESO Archive
(\href{http://archive.eso.org}{\bf
  \underline{http://archive.eso.org}}), and if so, to explain the
need for new data.

\subsection{Duplication of GTO/Public Surveys: {\bf BOX 9b}}

Specify whether there is any duplication of targets/regions covered by ongoing GTO and/or Public Survey
programmes. 
If so, please explain the need for the new data here. Details on the protected target/fields in these
ongoing programmes can be found at the following webpages for GTO programmes and Public surveys: \\
\href{http://www.eso.org/sci/observing/teles-alloc/gto.html}
{\bf  \underline{http://www.eso.org/sci/observing/teles-alloc/gto.html}}\\
\href{http://www.eso.org/sci/observing/PublicSurveys/sciencePublicSurveys.html}
{\bf \underline{http://www.eso.org/sci/observing/PublicSurveys/sciencePublicSurveys.html}}.

\subsection{Applicant's  Publications: {\bf BOX 10}}

The macro \verb|\Publications{}| should be used to 
provide a list of your publications related to the
subject of the current proposal and published during the past two
years. The A\&A simplified abbreviations for references should be
used. The individual references should be separated with a small
amount of vertical space, to be created with the standard \LaTeX\
command \verb|\smallskip\\|. For example:
\begin{verbatim}
   \Publications{
   Name1 A., Name2 B., 2001, ApJ, 518, 567: Title of article1
   \smallskip\\
   Name3 A., Name4 B., 2002, A\&A, 388, 17: Title of article2
   \smallskip\\
   Name5 A. et al., 2002, AJ, 118, 1567: Title of article3
   }
\end{verbatim}

\subsection{List of Targets: {\bf BOX 11}}
  
Provide the complete list of targets to be observed in this programme,
by using the macro \verb|\Target| 
with the following nine parameters: run identifier (you may use the same
target/field in more than one run), target field/name, Right Ascension
(hh mm ss.f, or hh mm.f, or hh.f) and Declination (dd mm ss, or dd
mm.f, or dd.f) for the 
J2000 equinox, requested time on target (in hours
with overheads and calibration included), magnitude, angular diameter,
additional information (see below), and reference star identifier
(see below)
for each target field.  Please use the format \{00~00~00\}
in case of unknown coordinates. There can be as many occurrences of
the macro \verb|\Target| as required to accommodate all targets of all
runs of the programme. Long lists of targets will continue
on the last page(s) of the proposal form.

{\bf Please note} that the scheduling of your programme will take
{\bf all targets} given in this list into account. {\bf Only include
 the targets requested for Period 100}.
%(except for VLT-XMM proposals).
Make sure your targets
are significantly observable during this period.  Inclusion of targets
with insufficient visibility during the current period may result in
rejection of your programme by the automatic scheduler.

The additional information field (8th argument of the \verb|\Target|
macro) may in general be used to provide any relevant piece of
information about the target that does not pertain to any other
argument of the macro (e.g. the period of a variable star). However,
for APEX targets, usage of this field is {\bf mandatory} to indicate the
requested Precipitable Water Vapour (PWV) and the acceptable range of
Local Sidereal Time (LST) for the considered observation. The format
should be similar to the one shown in the following example:
\begin{verbatim}
\Target{A}{HD 104237}{12 00 05.6}{-78 11 33}{1}{}{}{PWV<0.7mm;LST=9h00-15h00}{}
\end{verbatim}

A reference source identifier must be provided for all natural guide 
stars (NGS), in the case of NGS observations with NACO, SINFONI and CRIRES, 
all tip-tilt stars (TTS), in the case of all laser guide star (LGS)
observations with SINFONI, and all Fringe Tracking stars, in the case
of all GRAVITY observations in dual-feed mode.
For observations with the noAO modes of SINFONI and NACO,
%CRIRES, 
you do not need to provide this information.  {\bf The reference source
  designation has to be the exact identifier of the selected star
  either 
  from the Guide Star Catalog 2 (GSC2) or the 2MASS point source
  catalogue.} Note that GSC2 stars identifiers should NOT be preceded
by GSC2, 
but must start with either N or S. In case the reference source is not
included in either catalogue, for instance because it is a supernova or
a solar system object, ``alt'' should be entered as reference source
identifier, and additional information can be provided in the
\verb|\TargetNotes| macro.
Rules for reference star designation can be found for GSC2 at:\\
  \href{http://vizier.u-strasbg.fr/viz-bin/VizieR-n?-source=METAnot&catid=1271&notid=1&-out=text}{\bf
    \underline{http://vizier.u-strasbg.fr/viz-bin/VizieR-n?-source=METAnot\&catid=1271\&notid=1\&-out=text}}. 

\smallskip

Examples of valid and invalid GSC2 identifiers are given below:

\begin{center}
\begin{tabular}{ll}
%\multicolumn{2}{c}{} \\
%\hline
    N01230121           & good\\
    S33333331           & good\\
    n01230121           & bad\\
    N012301201          & bad\\
    S01230141           & bad\\
    S3333333000001      & good\\
    S01201201234567     & bad\\
%\hline
\end{tabular}
\end{center}


%\vspace{1cm}

For 2MASS, the rules for reference star designation are available at:\\
  \href{http://www.ipac.caltech.edu/2mass/releases/allsky/doc/sec2_2a.html}{\bf
    \underline{http://www.ipac.caltech.edu/2mass/releases/allsky/doc/sec2\_2a.html}}. 

\smallskip

Here are some examples of correct and incorrect identifiers:

\begin{center}
\begin{tabular}{ll}
%\multicolumn{2}{c}{} \\
%\hline
 2MASS J01234567+7801020      & good\\
 2MASS J00000000+7801020L     & good\\
 2MASS J01234567+9000000      & good\\
 2MASS J01234567+9000000W     & good\\
 2MASX J01234567+7801020      & bad\\
 2MASS J97234567+7801020      & bad\\
%\hline
\end{tabular}
\end{center}

Thus the following examples illustrate the correct usage of the
\verb|\Target| macro when a reference star must be specified:
\begin{verbatim}
    \Target{B}{NGC 105}{22 55 00}{-47 50 30}{9.0}{}{}{}{S33333331}
    \Target{C}{NGC 106}{00 24 43}{-05 09 00}{2.0}{}{}{}{2MASS J01234567+7801020}
\end{verbatim}

The macro \verb|\TargetsNotes{}| should be used to include any
comments that apply to several or all targets (or to specify reference
stars that are not found in the GSC2 or 2MASS catalogues).
\begin{verbatim}
   \TargetsNotes{The planned grid of pointings around the targets 
                listed above will be defined during the first 
                observing night.}
\end{verbatim}

\subsection{Scheduling Requirements: {\bf BOX 12}}
\label{sec:schedreq}
If your proposal involves any of the following:
\begin{itemize}
\item observations to be executed on specific dates (e.g., for
  simultaneity with observations at other facilities);
\item observations to be executed at pre-defined time intervals (e.g.,
  at different epochs so as to achieve phase coverage of a periodically
  variable target);
\item Visitor Mode runs split into non-consecutive nights (for which
  the fourth argument specifying the number of nights in Box~3 of the
  proposal form includes a formula);
\item Visitor Mode runs mutually linked, which need to be scheduled in
  a given sequence and at specified time intervals,
\end{itemize}
you {\bf must} uncomment the macro \verb|\HasTimingConstraints|.
Please leave the brackets in \verb|\HasTimingConstraints| blank.  Details of time
 constraints can be entered in Special Remarks and using the
 other flags in Box 12.

Please note that the macro \verb|\HasTimingConstraints| should 
be {\bf commented out} in the following cases:
\begin{itemize}
\item for scheduling constraints resulting only from the genuine
  visibility window of the target sources (defined by their location
  in the sky) or from the phases of the Moon; 
\item for time series of observations acquired during a single night
  or over several consecutive nights of a contiguous Visitor Mode run; 
\item for Target of Opportunity observations.
\end{itemize} 

In order to allow for the automated scheduling of all
Visitor and Service Mode observing runs, you must provide all
information related to the details of the way your programme should be
scheduled.  If you have any doubts about how to specify your particular
requirements please email OPO at \href{mailto:opo@eso.org}{\tt opo@eso.org}
well in advance of the proposal submission deadline.

\medskip

{\bf 1. RUN SPLITTING.} For Visitor Mode runs: if the fourth argument
you have provided in 
Box~3 indicates a simple number of consecutive nights or half-nights,
e.g.:
\begin{verbatim}
    \ObservingRun{A}{100}{FORS2}{4}{nov}{d}{0.8}{CLR}{v}{}
\end{verbatim}
you do not have to do anything.  If the fourth argument in Box~3
indicates a more complicated requirement, with some non-consecutive
nights, e.g.: 
\begin{verbatim}
    \ObservingRun{A}{100}{FORS2}{8n=3x2+4H2}{nov}{d}{0.8}{CLR}{v}{}
\end{verbatim}
you must provide some additional information.  The fourth argument above,
namely, \verb|8n=3x2+4H2|, means that you request a total of 8 nights,
made of three sub-runs of 2 consecutive nights each, followed later by
a sub-run of 4 consecutive second halves of a night.

You should use the macro \verb|\RunSplitting{}{}|, and put
the run identifier in the first argument.  The second argument should indicate
the way the run should be split into different sub-runs.  If the
interval between two sub-runs has to be exactly a given number of
days, say 20, then this is a {\bf strong} constraint and this number
of days should be followed by the letter \verb|s|.  If the interval
between two sub-runs has some tolerance on the number of days, say
20$\pm$5, then this is a {\bf weak} constraint and this number of days
should be followed by the letter \verb|w|.  Consequently, if
\verb|8n=3x2+4H2| means that you want three sub-runs of 2 consecutive
nights each, the first and the second separated by 10 nights exactly,
the second and the third separated by more or less 20 nights, followed
exactly 15 days later by one sub-run of 4 consecutive second halves of
a night, then the second argument of the macro
\verb|\RunSplitting{}{}| should contain the following expression:
\verb|2,10s,2,20w,2,15s,4H2|. Hence the following entry should appear
in your proposal file:
\begin{verbatim}
    \RunSplitting{A}{2,10s,2,20w,2,15s,4H2}
\end{verbatim}

Note that it is necessary to specify explicitly the time intervals
among all non consecutive sub-runs.

The macro \verb|\RunSplitting| is meaningless for Service Mode runs,
for which constraints of the considered type should be fully specified
at Phase~2 (but the macro \verb|\HasTimingConstraints| should be
uncommented for such runs, so that they are duly flagged). {\bf
  However please note that in Service Mode, 
  monitoring programmes are executed on a best effort 
  basis only.} In particular, a monitoring sequence can be interrupted
by unsuitable weather conditions or by runs scheduled in Visitor Mode.

\medskip

{\bf 2.  LINK FOR COORDINATED OBSERVATIONS.} If you have requested
three different runs in Box 3, e.g.:
\begin{verbatim}
    \ObservingRun{A}{100}{FORS2}{2n}{nov}{d}{0.8}{CLR}{v}{}
    \ObservingRun{B}{100}{FORS2}{3n}{nov}{d}{0.8}{PHO}{v}{}
    \ObservingRun{C}{100}{UVES}{20h}{nov}{d}{0.8}{CLR}{v}{}
\end{verbatim}
and would like some of them to be simultaneous and some later than
others, independently of the exact period of scheduling, then use
\verb|simultaneous, after| and the macro \verb|\Link{}{}{}{}| in the
following way:
\begin{verbatim}
    \Link{B}{after}{A}{10}
    \Link{B}{after}{A}{}
    \Link{B}{simultaneous}{C}{}
\end{verbatim}

\medskip

{\bf 3. UNSUITABLE PERIOD(S) OF TIME.}  If you have requested two
nights in Box 3 and would like them to be scheduled to avoid some
unsuitable periods of time, for some reason, then use the macro
\verb|\UnsuitableTimes{}{}{}{}| in the following way:
\begin{verbatim}
    \UnsuitableTimes{A}{15-jan-18}{18-jan-18}{International Conference}
    \UnsuitableTimes{B}{15-jan-18}{18-jan-18}{International Conference}
    \UnsuitableTimes{C}{15-jan-18}{18-jan-18}{International Conference}
    \UnsuitableTimes{C}{1-jan-18}{3-jan-18}{Committee Meeting}
\end{verbatim}

Dates correspond to 12:00 noon Local Time at
the Observatory location (i.e., in Chile).
In other words the first date refers to the start of the first night
of the unsuitable period; the second date refers to the end of the last night.
As with the \verb|TimeCritical| macro, only one run can be specified in each
\verb|UnsuitableTimes macro|. 
Further explanation can also be entered in  \verb|\SpecialRemarks{}| (Box 5).

\subsection{Scheduling Requirements contd.. : {\bf BOX 12}}
\label{sec:timecrit}

{\bf 4. SPECIFIC DATE(S) FOR TIME CRITICAL  OBSERVATIONS.} If you have
requested 2 nights in Box 3, e.g.:
\begin{verbatim}
    \ObservingRun{A}{100}{FORS2}{2n}{nov}{d}{0.8}{CLR}{v}{} 
\end{verbatim}
and if for some reason (e.g., specific phase of a variable object
or parallel observations with already scheduled HST observations,
etc.)  you need these two nights scheduled between some specific
dates, then use the macro \verb|\TimeCritical{}{}{}{}| in the
following way:
\begin{verbatim}
    \TimeCritical{A}{12-nov-17}{14-nov-17}{parallel observation with HST}
    \TimeCritical{A}{20-nov-17}{25-nov-17}{co-ordinated with XMM}     
    \TimeCritical{B}{20-nov-17}{25-nov-17}{co-ordinated with XMM}
    \TimeCritical{C}{20-nov-17}{25-nov-17}{co-ordinated with XMM}
\end{verbatim}

Please note the following points:
\begin{itemize}
\item As of P87, the constraints set by this macro are checked against the
time specified by the \verb|\ObservingRun| in more detail.
Some examples of correct and incorrect uses of this pair of macros are shown below.
\item The indicated dates correspond to 12:00 noon LOCAL Time at
the Observatory location (i.e., in Chile). In other words, the first
date refers to the start of the first night of the acceptable
interval, and the second to the end of the last night. Please 
convert event times from Universal Time to Local Time.
\item Only one run can be specified in each \verb|TimeCritical| field. 
However, as an observing run can have multiple Time critical dates, users
can specify multiple separate \verb|\TimeCritical| macros for a single observing run (as shown above
for Observing Run A).
\item ESO can still access the information for this macro (and all other macros in Box 12) so
do not worry if the constraints overfill the available space.
\end{itemize}

{\bf Examples of correct usages of the \verb|\TimeCritical| macro: }\\
If the user requires an observing run to be split into three observations, each to taken in the 
second half of the night over a specific 8-day period, this could be specified using the following macros:
\begin{verbatim}
    \ObservingRun{A}{87}{VIMOS}{1.5n=3H2}{jun}{n}{0.6}{CLR}{v}{}
    \RunSplitting{A}{0.5,4w,0.5,4w,0.5}
    \TimeCritical{A}{12-oct-17}{20-oct-17}{reason here}
\end{verbatim}

Conversely if the ObservingRun was specified as above but the TimeCritical macro was configured as follows
it would be rejected as the configuration 1.5n=3H2 implies there must be a minimum time separation of
three nights.
\begin{verbatim}
    \TimeCritical{A}{12-oct-17}{14-oct-17}{reason here}
\end{verbatim}








\subsection{Instrument configuration: {\bf BOX 13}}

The two template proposals ({\tt template.tex} for normal
applications, 
and {\tt templatelarge.tex} for Large Programme applications) contain
the full list of configurations for all available instruments at all
available ESO telescopes (Paranal, La Silla and Chajnantor).  In order
to provide 
general information about the setup of the ESO instrument(s) you plan
to use, please uncomment only the lines related to the instrument
modes and configurations needed for the acquisition of your desired
observations.  For some lines related, e.g., to special filters or
central wavelength, please add the required information where
appropriate (between the already existing curly
braces).

Note that you {\bf must} put the run ID within the first pair of curly
braces of the relevant lines. {\bf Do not} specify any instrument
configuration for alternative runs or visitor instruments (see Box 3). 
Note
that all parameters are {\bf mandatory} for the \verb|\INSconfig|
macro (do not use empty fields).

\subsection{Interferometry page}
\label{sec:vlti}

For VLTI runs, a separate run must be specified for each requested baseline configuration.
If your proposal includes VLTI runs, you MUST uncomment and fill in the arguments of the 
macro \verb|\VLTITarget| with run ID, target name, visual
magnitude, magnitude at wavelength of observation, wavelength of
observation (in microns), size at wavelength of observation (in mas),
baseline (see the following website for available configurations:
\href{http://www.eso.org/paranal/insnews/vlti_overview.html}{\bf
  \underline{http://www.eso.org/paranal/insnews/vlti\_overview.html}}), 
range of visibilities for the specified
configuration (at preferred hour angle or at hour angle 0), correlated
magnitude, and time on target (ToT) in hours. For example,
\begin{verbatim}
\VLTITarget{E}{NGC 106}{-0.7}{-3.5}{10.6}{40}{UT1-UT2-UT3}{0.84}{-2.5}{6}
\VLTITarget{F}{NGC 107}{-0.7}{-3.5}{2.1}{40}{UT1-UT2-UT3}{0.84/1.0/0.1}{1./0.5/2.}{6}
\VLTITarget{G}{NGC 108}{-0.7}{1.5}{2.1}{40}{D0-K0-G2-J3}{0.90/0.80/0.70}{1.6/1.7/1.9}{6}
\end{verbatim}


For UT observations, please specify one of the available VLTI baselines.

Please specify one of the available AT quadruplets listed in the
template proposal files. At Phase 2, the time can then be split
among the AMBER triplets of the chosen quadruplet.

For observations with AT baselines, please use one typical baseline of the
quadruplet that you have specified in order to compute
representative visibility and correlated magnitude values.
Please specify the maximum and minimum visibility and correlated flux values
corresponding to the chosen configuration at hour angle 0 separated by 
a slash "/".
Similarly, the correlated magnitudes for the various baselines are also
specified as three values separated by a slash.

You can use the macro \verb|\VLTITargetNotes| to insert comments about
some or all of your VLTI targets. You should take advantage of this
macro to indicate suitable alternative baselines for your
observations.

\subsection{ToO page}
\label{sec:toopage}


If your programme has Target of Opportunity runs (ToO) runs, 
you should specify these in the  \verb|\ObservingRun| macro 
using ``TOO'' in the corresponding field. 
Please see Section 4.4 of the Call for Proposals for more information on how ToO runs should be defined.
You will also find more details on which settings are available for ToO runs on specific instruments.

In the ESOFORM the ToO information must be filled in for the run
using the corresponding   \verb|\TOORun| macro.  The arguments in this macro 
are, in order: the run identifier,
the nature of the observation, the number of 
targets per run, and the number of triggers per targets. There must be
one occurrence of the macro \verb|\TOORun| for each of the ToO runs specified
in Run Type field in  the \verb|\ObservingRun| macro (Box~3).
 The second argument (nature of the observation) may be one
of the following keywords:
\begin{itemize}
\item RRM, for observations to be triggered via the automated Rapid
  Response Mode system within 4 hours after an event;
\item ToO-hard, for observations for manually triggered observations that must
  be carried out as soon as possible or at most within 48 hours of receipt of the trigger by the
  Observatory (and in most cases, as soon as possible), or observations that
  involve a strict time constraint (i.e., that must be executed during
  a specific night);
\item ToO-soft, for manually triggered observations for which the
  Observatory can receive notification more than 48 hours before
  execution, and which can be scheduled for execution with a
  flexibility of at least $\pm1$\,day.
\end{itemize}
Only one keyword can be specified for each run. If observations
pertaining to different categories are needed,
several runs must be defined. The number of triggers must be indicated
for RRM, ToO-hard and ToO-soft observations only.
An occurrence of the macro \verb|\TOORun|
looks like the following example:
\begin{verbatim}
\TOORun{A}{ToO-hard}{2}{3}
\end{verbatim}

You have the opportunity to add some notes to the ToO page by using
the macro \verb|\TOONotes|.
As a rule, ToO proposals should involve at least one trigger
of one of either the RRM or ToO-hard or ToO-soft types.



\subsection{The Visitor Instrument Page}
\label{sec:visins}

The following commands are only needed for proposals involving a Visitor
Instrument, in which case they are also {\bf mandatory}.  You should
uncomment them and provide the required information between the
different pairs of curly braces. Please do not enter anything in Box 13 
(i.e. \verb|\INSconfig|) for Visitor Instrument runs.

If applying for the VLTI visitor instrument, SpecialVLTI, please remember to uncomment
and fill in the \verb|\VLTITarget| macro and this Visitor Instrument page.

\begin{verbatim}
    %\Desc{}   % Description of the instrument and its operation
    %\Comm{}   % On which telescope(s) has instrument been commissioned/used
    %\WV{}     % Total weight and value of equipment to be shipped
    %\Wfocus{} % Weight at the focus (including ancillary equipment)
    %\Interf{} % Compatibility of attachment interface with required focus
    %\Focal{}  % Back focal distance value
    %\Acqu{}   % Acquisition, focusing, and guiding procedure
    %\Softw{}  % Compatibility with ESO software standards (data handling)
    %\Suppl    % Estimate of services expected from ESO (in person days)
\end{verbatim}


\section{HOW TO FILL A LARGE PROGRAMME TEMPLATE}
\label{sec:large}

The ESOFORM package includes a specific template ({\tt
  templatelarge.tex}) that must be used to generate a Large Programme
proposal and GTO programmes that run over two periods. This template is only slightly different from the
template for Normal Programmes. Hereafter is the list of the
differences between the two templates. Any feature not appearing in
this list is {\bf identical} in both templates though the box number
may differ.

\begin{table}[t]
\caption{Keywords of Available Instruments (Large Programmes)}
\label{tab:inslarge}
\medskip
\begin{center}
\begin{tabular}{@{\extracolsep{0pt}}l@{\extracolsep{40pt}}l@{\extracolsep{0pt}}}
\hline
\hline \\[-6pt]
Telescope & Instrument keywords     \\[4pt]
\hline                              \\[-6pt]
UT1       & FORS2, KMOS\\
%, NACO$^1$    \\
UT2       & FLAMES, UVES, XSHOOTER\\
UT3       & SPHERE\\% VISIR$^1$ VIMOS
UT4$^1$   & HAWK-I, MUSE\\
VLTI$^2$  & GRAVITY, PIONIER\\
VST       & OMEGACAM$^4$\\
NTT       & SOFI, EFOSC2, SpecialNTT\\
3.6       & HARPS, Special3.6       \\
APEX      & ARTEMIS, SEPIA$^5$ \\
\hline
\end{tabular}
\end{center}
$^1$ Large Programmes for UT4 instruments must be compatible with
constraints imposed by the UT4 activities described in
the Call for Proposals, taking into account that the schedule
described is provisional.\\ 
$^2$ Large Programmes on the VLTI must be compatible with the
VLTI and (if relevant) UT4 activities described in the Call for Proposals, 
taking into account that the schedule
described is provisional.\\
$^4$ OMEGACAM is only available for programmes that request poor
weather conditions (see the Call for Proposals).\\
$^5$ Both the band 5 and band 9 receivers are offered.\\
\end{table}


\begin{itemize}
\item The \verb|\GTOcontract| macro should only be commented out and filled in if submitting 
a GTO proposal that requests over 100 hours. As described in the Call for Proposals these can run over two periods. The Programme type must always be {\tt LARGE} for all proposals
prepared using the {\tt templatelarge.tex} template. 
\item Box 2: the abstract for Large Programme proposals (and Large GTO programmes) can be
  slightly longer (up to 13 lines) than for Normal Programme proposals
  (up to 9 lines).  
\item Box 3: proposers must specify {\bf all the runs across all the
    Periods} covered by the Large Programme. 
For La Silla and Paranal instruments this means a maximum of 4 periods, while for APEX
instruments only up to Period 100 is allowed (see the Call for Proposals). 
Note that regardless of the instrument, Large GTO Programmes can only run up to two periods.
A Large (and Large GTO) Programme can have up to 26 runs. Since the space for the runs
  description in Box 3 is limited to 12 lines, a new box containing
  the observing runs beyond this limit will be created at the end of
  the proposal form if needed.
\item Box 3: the list of keywords of all instruments offered in Period
  100 for Large (and Large GTO) Programmes is given in
  Table~\ref{tab:inslarge}.
\item Box 5: while two pages (including figures) are allowed for the scientific
  description of Normal Programmes, the
  description of Large (and Large GTO) Programmes may take up to three pages followed by
  two additional pages for figures.

  The scientific description is comprised of the following four subsections:
  \verb|\ScientificRationale{}|,
  \verb|\ImmediateObjective{}|, \verb|\TelescopeJustification{}|, and\\
  \verb|\ModeJustification{}|.  
  The \verb|\TelescopeJustification{}| and \verb|\ModeJustification{}|
  macros are the same as those in Boxes 8b and 8c in the Normal Programme
  template.
  
  As noted above there are also there are two additional pages for
  attachments (e.g., figures, etc). 
\item Boxes 6 and 7: specific to the Large (and Large GTO) Programme proposals, these
  two boxes should contain the required information describing the
  experience of the applicants with telescopes, instrumentation, 
  data reduction  and delivery of data products to the ESO Archive 
  (\verb|\Experience{}|). The strategy for data reduction, analysis 
  and a delivery plan for data products into the ESO Archive should 
  be presented along with a brief description of the resources available
  to the observing team, such as: computing facilities and research
  assistants (\verb|\Resources{}|).
\item Box 8: Additional remarks or comments can be provided using
  the macro \verb|\SpecialRemarks{}|.
\end{itemize} 



%%%%%%%%%%%%%%%%%%%%%%%%%%%%%%%%%%%%%%%%%%%%%%%%%%%%%%%%%%%%%%%%%%%%%%

\section{SUBMISSION OF THE APPLICATION}
\label{sec:submission}

Proposals must be prepared as pdf\LaTeX\ source files, making use of
the {\bf latest ESOFORM} package, corresponding to the ESO Period
for which they are submitted.  Proposals received in
any other format, or with modified ESOFORM macros, will be
automatically rejected by the automated proposal handling system.

When the \LaTeX\ source file of your application is complete, {\bf
  please process it with pdf\LaTeX} so as to identify any
possible \LaTeX\ format errors. In particular, we {\bf strongly}
recommend that you
\begin{itemize} 
\item review the log file generated by \LaTeX\ so as to
  check for the presence of warning messages issued by the ESOFORM
  macros. Such messages report, among others, instances in which a text
  field is too long, so that your input is truncated in the pdf file
  that is generated, and part of the information that you
  submit will not be communicated to the OPC;
\item carefully inspect a printed copy of the output to make sure that
  all parts of the application are completed, and that their
  formatting is appropriate.
\end{itemize} 

{\bf Please note} that while several checks are
performed by the ESOFORM package when running pdf\LaTeX, successful 
compilation  {\bf does not guarantee} that a proposal is
fully compliant. Indeed, many key  checks 
can only be performed by the proposal reception system after the proposal
has been uploaded within the ESO User Portal.

{\bf You should verify that your proposal complies with ESO
requirements using a ``skeleton'' version of  your proposal that
only contains the technical details of your  programme. This should be done well before the Phase 1 Deadline.}
 Please upload the ``skeleton'' proposal as if you were submitting it and follow the instructions online.
If the proposal passes all the technical checks you will get the message:
``Your proposal is verified and passed all the checks (but not submitted yet)!
''. You can then either log out of the User Portal or 
return to the submission page by selecting the option to 
``Go back to the beginning''. 
More details on the full proposal submission procedure are given below.

Proposals are submitted by logging into the ESO User Portal.
In order to do this, proposers must be registered and have activated
their User Portal accounts. To submit please go to:
\begin{center}
  \href{http://www.eso.org/UserPortal}{\bf \underline{http://www.eso.org/UserPortal}}
\end{center}
Once you have logged in, Once you have logged in, you should select the item called ``Download ESOFORM packages'' under the Phase 1 menu items.  In order to submit, select ``Submit a proposal'' from the same menu. You will then be taken to the WASP page (Web Application for Submitting Proposals). 
On choosing the relevant cycle, you will be asked to upload the \LaTeX\ file of your proposal and should follow the subsequent instructions. \\
A number
of checks are executed at the various steps of the submission process;
please follow the instructions online. If a problem is detected 
it will be clearly reported by the system: fix it in your proposal and
resubmit it.
Once the proposal passes through all the checks, you will be requested to finalise the
submission by clicking on the corresponding button. {\bf It is
  essential that you execute this final step}: your proposal will not
be submitted until this is done, even though you have uploaded all the
necessary files! \\
 Upon submission of
a correctly completed proposal, the ESO proposal validation
software will return an identifier assigned to the valid proposal. This
identifier, and the acknowledgment page in which it appears, represent
the official confirmation that the proposal successfully entered the
proposal handling system. We recommend that you take note of the
identifier; you may also want to print the acknowledgment page for
your records. In addition, an email confirmation
is sent to the submitter and to the PI of the proposal.


Some common problems are described below.
\begin{itemize}
\item BibTeX formats are not permitted within the ESOFORM package.
\item Figures without the corresponding .jpg or .pdf extensions will make the receiver hang.
\item The proposal reception system checks for {\bf the presence of text outside
 the argument fields of the ESOFORM macros} in the \LaTeX\ source of
the proposal, and rejects proposals in which such text is
found. If there is text outside the macros it appears as text or extra space above 
the ESO logo on top of the first page.  
This input {\bf must be commented out or relocated within the relevant macro} before the proposal is
submitted. 

\end{itemize}




\subsubsection*{Submission Problems} 
The proposal submission
acknowledgment page normally appears within seconds of completion of a
a submission. However, during the last few hours before the Phase 1
deadline, the system may be slowed down by the high load, and the
acknowledgment process may take several minutes. Please {\bf be
  patient}: even though it may look like ``nothing is happening'', the
system most likely is actually busy processing a queue of
proposals. Please {\bf do not} abort your submission or make a new
attempt at submitting the same proposal as this would only increase the
load on the system and make it slower. 

If you have not succeeded in completing your submission within 1~hour, 
please contact ESO via email  
at \href{mailto:esoform@eso.org}{\tt esoform@eso.org}.
{\bf Do not under any circumstance} attach your application (in any
format) to this email. {\bf Do not try to resubmit your application}
before receiving further instructions from ESO. {\bf Do not panic!}
Once you have uploaded the \LaTeX\ source of your proposal, your
attempt, and the time at which you initiated it, are recorded in the
ESO system, so anomalous delays due to the proposal reception
system will be duly identified. {\bf Be aware} that if you experience
difficulties due to the proposal reception system, you are most likely 
not the only user in this situation. As problem reports must be
handled sequentially it may take some time before you receive
feedback from ESO.  

As mentioned above, the acknowledgment Web page providing the
identifier of your proposal is the official confirmation of its
successful submission. The subsequent email notification is only sent
to you as a secondary confirmation, and delay in its delivery should
not represent a concern. However, if you have not received
it within 24 hours of your submission, please report this anomaly to  
\href{mailto:esoform@eso.org}{\tt esoform@eso.org}. 

A safe way to avoid submission problems (often related to heavy
system load  during the last few hours before the deadline) is to submit your
proposal early.  We strongly encourage you to send in your
applications and all attachments several days before the deadline.
  
\vspace*{1cm} {\bf All proposals and their attachments must reach the
  ESO servers via the WASP interface BEFORE 12:00 noon (Central
  European Summer Time) on the date of the deadline. Please note that  
  ESO cannot provide support beyond 11:00 on that day. 
  Responsibility for 
  verifying that ESO has correctly received, processed, and
  acknowledged your proposal before the proposal submission deadline
  rests entirely with you.  Revisions, corrections, and/or
  modifications will not be accepted after the deadline.}

\vspace*{1cm}

\begin{center}
\framebox{
\parbox[t]{135mm}{
  \noindent
  \begin{center}
    {IMPORTANT NOTICE}
  \end{center}
  Electronic proposal submission does not allow applicants to sign
  their proposals. Therefore ESO assumes that PI's take full
  responsibility for the contents of the proposal, in particular in
  regard to the names of co-investigators and the agreement to act
  according to the instructions for visiting astronomers, should
  observing time be granted.}}
\end{center}

\end{document} 

